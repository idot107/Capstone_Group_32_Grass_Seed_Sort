\documentclass[letter,draftclsnoffot, onecolumn]{IEEEtran}
\usepackage[singlespacing]{setspace}
\usepackage[margin=0.75in]{geometry}

\title{Problem Statement of Computer Version Grass Seed Sorter }
\author{Haoxuan Zhang }
\usepackage{natbib}
\usepackage{graphicx}
\IEEEspecialpapernotice{CS 461 Fall 2018}

\begin{document}

\maketitle

\section{Abstract}
Our team is assigned to a project called Computer Vision System of Grass Seed Sorter. The goal of the project is to create a computer device to separate off-type seeds from pure seed, and the technology will be used to view thousands of seeds/hr.  The basic process of how the machine works is that the seeds will be presented under the high resolution-camera while traveling down a vibratory conveyor. A LED light will indicate when an off-type seed or plant material has been spotted. Our main job is to use the latest equipment available to develop a computer vision system that will enhance the work performed by seed analysts by pre-screening the seed samples. Our final goal is to correctly classify 2500 seeds in 30 minutes. The purity of the target seed component must be 99.5 percentage, but a false negative rate of up to 10 percentages is acceptable. The entire project has to be fully transferable to the client.
\pagebreak

\section{Definition and description of the Problem}
Our team is assigned to a project called Computer Vision System of Grass Seed Sorter. The goal of the project is to create a computer device to separate off-type seeds from pure seed, and the technology will be used to view thousands of seeds/hr.  The basic process of how the machine works is that the seeds will be presented under the high resolution-camera while traveling down a vibratory conveyor. A LED light will indicate when an off-type seed or plant material has been spotted. Our main job is to use the latest equipment available to develop a computer vision system that will enhance the work performed by seed analysts by pre-screening the seed samples. Our final goal is to correctly classify 2500 seeds in 30 minutes. The purity of the target seed component must be 99.5 percentage, but a false negative rate of up to 10 percentages is acceptable. The entire project has to be fully transferable to the client.
\par
To make the machine vision system problem easier, it would be helpful to get rid of some seeds that have significant differences with the target seed. After the computer vision system identifies the off-type seed, how can we make the mechanical separator or robotic arms precisely dispose of these off-type seeds among pure target seeds? Our goal is to make the target seed component 99.5 percentage after the machine separates off-type seeds and target seeds. We need to know the specific number of off-type seeds that were originally in the test sample, and the number of off-type seeds that were rejected, then we will be able to get the percentage purity of target seed component. The problem is making the machine record the number of off-type seeds that are rejected or identified by the computer vision system. We are currently trying to solve these problems.
\pagebreak

\section{Proposed solution}
We plan to add two vibrating sifters to let the test sample go through. The first Vibrating sifter will have holes smaller than the target seeds. It will drop off the tiny size of off-type seeds and keep pure seeds and off-type seeds that have bigger size. Then it will let the rest of seeds pass the second vibrating sifter that has bigger holes. The second vibrating sifter will keep the bigger size of off-type seeds and off-type seeds that have sizes similar to the target seed. The target seed will be dropped off on the conveyor and transferred to the machine to be identified by the computer version system. Adding two opposite functional vibrating sifters can eliminate the off-type seeds that are much bigger or much smaller than the target seed. This step will reduce the burden of identifying off-type seeds for the high-resolution camera and computer vision system. The high resolution of the picture is necessary for this project. The images that are taken by the high-resolution camera need to contain details of the seed’s surface, such as texture, shape, and color. The computer vision system needs to analyze and classify seeds from pre-screened seed samples. To do this, we first need to build a database that contains enough images of target seeds for the computer vision system to match images from pre-screening seed samples with the off-type seeds. The computer vision system will compare the images of the database with the pre-screened seed samples by matching the similarity of texture, color, and shape of images. If the match percentage achieved is higher than the threshold, then the seed is acceptable. Otherwise, a LED light will indicate this is an off-type seed or plant material. Our group also needs to develop a working model that delivers an electronic signal to the mechanical separator or robotic arm when an off-type seed has been identified. Once a off-type seed is identified, an LED will turn on, and an electronic signal will be sent to a mechanical separator or robotic arm, which will then dispose of these identified off-type seeds. Others will develop the mechanical separator or robotic arm that accepts the yes/no signal. We will determine the proportion of target seeds and off-type seeds in test sample, the purity of target seed component = (pure seed number) / (rest of seed number).\pagebreak

\section{Performance metrics}
The matrix to determine if our team completed the project or not will contain the proportion of target seeds in the final sorted seeds. This proportion must be above 99.5 percentage and the sorter must be able to sort 2500 seeds in 30 minutes for the project to be considered successful. After a sample run, there will be two fractions of seed, the pure seed and the off-types. The purity of the target seed component must be 99.5 percentage. The purity of the off-type seeds must be greater than 90 percentage. Rejecting a small number of target seeds is acceptable since the vibrating sifters may reject part of the unsorted target seeds because they are too small or too large. At the end of the project, the system must be fully transferable for the client to use, including all equipment necessary to run the program, developed libraries, and instructions on how to train the system.
\bibliographystyle{IEEEtran}
\end{document}
