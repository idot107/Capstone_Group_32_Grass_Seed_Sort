\documentclass[onecolumn, draftclsnofoot,10pt, compsoc]{IEEEtran}
\usepackage{graphicx}
\usepackage{url}
\usepackage{pgfgantt}
\usepackage{setspace}
 \usepackage{geometry}
\geometry{textheight=9.5in, textwidth=7in}
 % 1. Fill in these details
\def \CapstoneTeamName{		The Seed Team}
\def \CapstoneTeamNumber{		32}
\def \GroupMemberOne{			Bharath Padmaraju}
\def \GroupMemberTwo{			Kevin Deming}
\def \GroupMemberThree{			Haoxuan Zhan}
\def \GroupMemberFour{			Cong Yang}
\def \GroupMemberFive{			Christopher Wohlwend}
\def \CapstoneProjectName{		Pure Grass Seed Sorter}
\def \CapstoneSponsorCompany{	Oregon State University Seed Lab}
\def \CapstoneSponsorPerson{		Dan Curry}
 % 2. Uncomment the appropriate line below so that the document type works
\def \DocType{	%	Requirement Document
				%Requirements Document
				Technology Review
				%Design Document
				%Progress Report
				}
			
\newcommand{\NameSigPair}[1]{\par
\makebox[2.75in][r]{#1} \hfil 	\makebox[3.25in]{\makebox[2.25in]{\hrulefill} \hfill		\makebox[.75in]{\hrulefill}}
\par\vspace{-12pt} \textit{\tiny\noindent
\makebox[2.75in]{} \hfil		\makebox[3.25in]{\makebox[2.25in][r]{Signature} \hfill	\makebox[.75in][r]{Date}}}}
% 3. If the document is not to be signed, uncomment the RENEWcommand below
\renewcommand{\NameSigPair}[1]{#1}
 %%%%%%%%%%%%%%%%%%%%%%%%%%%%%%%%%%%%%%%
\begin{document}
\begin{titlepage}
    \pagenumbering{gobble}
    \begin{singlespace}
    	\includegraphics[height=4cm]{coe_v_spot1}
        \hfill 
        % 4. If you have a logo, use this includegraphics command to put it on the coversheet.
        %\includegraphics[height=4cm]{CompanyLogo}   
        \par\vspace{.2in}
        \centering
        \scshape{
            \huge CS Capstone \DocType \par
            {\large\today}\par
            \vspace{.5in}
            \textbf{\Huge\CapstoneProjectName}\par
            \vfill
            {\large Prepared for}\par
            \Huge \CapstoneSponsorCompany\par
            \vspace{5pt}
            {\Large\NameSigPair{\CapstoneSponsorPerson}\par}
            {\large Prepared by }\par
            Group\CapstoneTeamNumber\par
            % 5. comment out the line below this one if you do not wish to name your team
            \CapstoneTeamName\par 
            \vspace{5pt}
            {\Large
            %    \NameSigPair{\GroupMemberOne}\par
            %    \NameSigPair{\GroupMemberTwo}\par
            %    \NameSigPair{\GroupMemberThree}\par
            %    \NameSigPair{\GroupMemberFour}\par
                \NameSigPair{\GroupMemberFive}\par
            }
            \vspace{20pt}
        }
        \begin{abstract}
        % 6. Fill in your abstract    
        	The primary objective of the project is to automate grass seed sorting. The members of the group will be building software to be able to discriminate between pure grass seeds from all other plant seeds including but not limited to weeds, and crop seeds. The method we will utilize will be a combination of implementing computer vision and deep learning algorithms to accurately identify off type seeds under a high definition camera. This will vastly reduce the stress and workload imposed upon seed analysts, and likely speed up the sorting process. Not only does this project offer a opportunity to improve seed research, but also creates possibilities in other fields where our technology can automate menial and repetitive tasks.
        \end{abstract}     
    \end{singlespace}
\end{titlepage}
\newpage
\pagenumbering{arabic}
\tableofcontents
% 7. uncomment this (if applicable). Consider adding a page break.
%\listoffigures
%\listoftables
\clearpage
 % 8. now you write!

\section{Database}
Our system will require a database to store training data for our neural network. This will primarily  be used during development and testing, but will still be an immensely important part of that process. Our databases will like have a couple thousand entries, but those entries will only have 2-4 columns of data. Our criteria are reliability, speed, project accessibility, and manual accessibility. 
\subsection{SQLite}
SQLite is likely the simplest of these solutions. Looking at the tutorial, SQLite provides a simple, and easy to use interface when working with c or c++ programs.\cite{sqlite} SQLite doesn't utilize a separate server to process transactions, but has a speed that is often faster then typical file access such as utilizing fopen().\cite{sqlite} If choosing a language other then C/C++, we may have to put extra effort into interfacing with the system, but manually executing SQL queries is simple. Finally SQLite is open source and developed full time by a team dedicated to building a reliable program, with tests providing 100\% branch coverage.\cite{sqlite}
\subsection{PostgreSQL}
PostgreSQL looks to be similarly reliable to SQLite, although also has a larger footprint in terms of setup. By using libpq, we would receive a clean interface to the database using c or c++. \cite{libpq} Additionally, a graphical interface known as pgAdmin is available to easily work with, view, and edit the database for Windows systems. \cite{pgadmin} PostgreSQL will run on all major operating systems, and has had over 30 years to work out kinks and bugs. \cite{postgres}
\subsection{CUBRID}
CUBRID is not available on Apple systems, although this is unlikely to be an issue as we are unlikely to use one. CUBRID contains easy to follow installation instructions, as well as an interface, CUBRID Manager, to make working with the database easier. \cite{cubrid} CUBRID also contains a handy Java interface in the event we choose to utilize the Java programming language through JDBC. \cite{cubrid} While it looks slightly less powerful then something like PostgreSQL, our use cases are simple, and CUBRID looks to be a little more intuitive.
\subsection{Chosen Solution}
After looking at the variety of solutions available, SQLite looks to be the best. Having a lightweight and server-less solution allows us to easily handle it on any device. It has a fast speed that makes it easy to automatically store new entries at run time. PostgreSQL and CUBRID both look like powerful tools, but many of their features would be left unused, leaving a lightweight solution extremely appealing. With most of our members familiar with C/C++, we are likely to make use of the simple C/C++ interface provided.

\section{Neural Net Storage}

While any computer with enough memory is capable of evaluating a neural net function, depending on the amount of RAM, and optimization, the cost in time could suffer. If using a lightweight computer such as a raspberry pi, it would be unfeasible to run in the time frame necessary, which is a little over 10 seeds per second. Solutions to this problem can include connecting to a server on a separate device or utilizing a product such as Intel's Movidius Neural Compute Stick. 
\subsection{Built In Memory}
If we choose not to use any of the optimization below, simply using whatever device our main code runs on will be our solution. This would likely be adequate, although it may slow down our final result, this is definitely the cheapest solution, as it imposes no additional cost to the project.
\subsection{Movidius Intel Neural Compute Stick}
The Movidius Neural Compute Stick is a product produced by Intel. This contains a Visual Processing Unit that is designed at a hardware level to accelerate neural network implementations, especially for computer vision. \cite{intel} This of course will have a very direct impact on the maximum number of images we can process, greatly increasing the maximum speed at which we can scan seeds. The Movidius Neural Compute Stick can be easily interfaced with by utilizing the Neural Compute API or SDK, which both give C, C++ and Python tools to assist with the development of a final application. \cite{intelintro} As of this writing, These compute sticks can be purchased for just under \$80, meaning that they are relatively affordable compared to our budget.

\subsection{Cloud Based Solution}
Utilizing a separate server will likely be necessary if we were to use a Raspberry Pi or similar lightweight computer, without a Neural Compute Stick. This solution does introduce constraints on network capabilities which could be problematic depending on the network capabilities of the seed lab. Further, we would need an actual server to run code on, or utilize a  service such as Google Cloud. This would introduce a recurring cost, but allows the server to be reused across multiple systems. Ultimately, while it is a potential solution, it introduces excessive extra setup, and likely costs more then a Compute stick in the long run.
\subsection{Chosen Solution}
Utilizing the Movidius is likely the most effective solution for our goals. It provides added efficiency at a low cost, while also providing a usable API for C/C++.

\section{Computer/Processor}
In order to coordinate all of the pieces of our system, we will need some kind of computer or processor at the center of it all. In the event that we want to build additional systems, choosing a cheap option would make copies of the system substantially cheaper. The computer will be responsible for UI, initial vision-processing, and communicating with the camera, neural network solution, and either storing or communicating with the database. Our two primary criteria then are cost, and capability at running those tasks.
\subsection{Raspberry pi}
The primarily appeal to the Raspberry Pi is its incredibly low cost. Available at the PiShop for only \$35.\cite{raspi}  If we are able to add the capabilities needed, such as using the Neural Compute stick above, this could be a cheap and effective solution. Their most recent model contains a single gigabyte of ram, and an interface for a display, camera, and several USB ports. \cite{raspi} Memory is lacking, and could be added for the database. The largest concern relates to the 1 GB RAM, which could struggle when pre-processing each image into images of individual seeds, depending on how fast this process is required to be. There is a sample tutorial available that utilizes a Raspberry Pi, and neural compute stick to perform some basic AI computer vision, so in concept this has everything we would need. \cite{piplusncs}
\subsection{Dell Inspiron i5577}
This laptop would contain all the basics needed to complete our task built in. This laptop has 16 GB of ram, and a Nvidia GTX 1050 4GB graphics card.\cite{laptop} It is touted as one of the cheapest laptops that is easily capable of delving into deep learning projects. \cite{laptop} While these capabilities are cool, the cost of \$950 is nearly equal to our total budget, and with external GPUs during training and neural compute sticks in the final product, utilizing such an expensive laptop is unlikely to effective results.
\subsection{The Laptop The Seed Lab Already Has}
The OSU Seed Lab already has a laptop available for our use. This has the advantage of being effectively free, while also providing enough capabilities such that we can prototype without fear of running into issues with total capabilities. While its capabilities aren't entirely known at this time, it should be capable of performing tasks not explicitly given to an external GPU, or compute stick.
\subsection{Chosen Solution}
Because we already have a laptop that the OSU Seed Lab can supply, we plan to utilize it to keep our capabilities open, and cost minimal. While it is likely that a Raspberry Pi is cheaper for future systems it is unclear how much extra help it would need to execute everything we need it to do.

\bibliographystyle{ieeetr}
\bibliography{bib}
\end{document}
