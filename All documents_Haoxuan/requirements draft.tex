\documentclass[10pt, letterpaper, twoside, draftclsnofoot, onecolumn. notitlepage]{article}
\usepackage[utf8]{inputenc}
\usepackage[margin=0.75in]{geometry}
\rmfamily

\title{Requirements draft}
\author{Haoxuan Zhang
CS 461 Fall Term}
%Senior Design Fall Term

\begin{document}
	\begin{titlepage}
		\maketitle
		\begin{abstract}
			\textit{Abstract} - For workers engaged in agriculture, they have high requirements for high purity seeds. Because the high-purity seeds will reduce unnecessary waste on the hybrids, higher yields can be picked regarding planting. In general, there is a dedicated staff sitting in front of the microscope to observe the seeds that are continuously being transported from the conveyor belt. They use they expertise knowledge of seeds to manually pick up the seeds. This amount of work is enormous for these workers, and the value made in this way are not proportional to the time they spend on. Therefore, it is necessary to develop a program that controls machine to do the job instead of human. From the client’s requirements for the product, the machine needs to complete the screening of 25000 seeds in 30 minutes and the purity of target seed must be above 99.9 percentage, which requires the program to be efficient in identifying off-type seeds. When the machine identifies the off-type seed, it should be able to locate the bad seed, and the bad seed will be picked up manually. This document will give a brief overview of the overall idea of the entire project and the technologies that may be used. The article will describe the principle and composition of the application program and the connection process between the program and the machine.
		\end{abstract}
	\end{titlepage}
\section{Introduction}
\quad This section describes the scope of seed project and states the purpose of the document which is to give some definitions of terms which will be used in the computer vision grass seed sorter project. 
\subsection{. Purpose:
The purpose of this document is to describe and explain the requirements of computer vision grass seed sorter project in details. The paper will inform readers about the technologies that will be used in the project and how the program connected with the machine. The report will also explain why the computer can identify off-type seed, which relates to the neural network system. }
\subsection{. Scope:
The computer vision grass seed sorter uses neural network system technology to let computer discriminate off-seeds from test sample instead of labor. After the project is done, it will be delivered to Oregon State University Seed Lab. The computer vision grass seed sorter will help the seed lab to save labor and lots of time on discriminate seed by manually.} 
\subsection{
. Definition
Neural network system: It is a computing system vaguely inspired by the biological neural networks that constitute animal brains.}
\subsection{
Test sample: 25000 seeds consisted by target seed and off-type seed
Target seed: It is the seed we want from test sample}
\subsection{
Purity: The percentage of target seed in test sample}
\subsection{
Off-type seed: It is the seed we want to get rid from test sample}
\subsection{
High-resolution: a display or a photographic or video image that shows a large amount of detail}
\subsection{
Electronic signal: an electric current that represents information}
\subsection{
Discriminate: to find off-type seed from test sample}





\end{document}