\documentclass{article}
\usepackage[utf8]{inputenc}

\title{Neural network_design document}
\author{zhanhaox }

\begin{document}

\maketitle

\section{Neural Network}
The specific structure of the human brain and eyes enable us to distinguish the things we see quickly. It is hard to make a software program to do the same thing by just using traditional programming methods. Neural networks solve this obstacle differently. The technique is to let the system through a training process with a large amount of experimental data to teach the neural network how to identify the target object. A critical component of the neural network is called perceptron. The perceptron absorbs multiple inputs and generates the result. For a perceptron, every input has a weight(threshold) that can represent the importance of that input to the output, and the perceptron can base on the weight to determine what the result will be. We can set different weights for input, then the decision-making models of perceptron will be different. What we just introduced is how a single perceptron works. If we connect multiple perceptrons, they will build a complex network of perceptron's that can make subtle decisions.  Each perceptron of the neural network only has one output, but the result is used as an input for many of other perceptrons. The weight or we can call it threshold condition judging is too heavy for the neural network system to do calculation; it will be better to use bias to simplify the notation of the condition. [1]
Six types of artificial neural networks that are used in machine learning currently: radial basis functions neural networks, feedforward neural networks, recurrent neural networks, convolutional neural networks, Kohonen self-organizing neural networks, and modular neural networks. The neural network we introduced above is called feedforward neural network – artificial neuron, and we will use this specific neural network in our computer vision grass seed sorter project to identify seeds.  [2]
The neural network system requires clear and informative data to train.  We also need to initialize our weights randomly. There is a function called loss-function that is used to determine how good our neural network performs in a particular task. In the process of training, we start with a lousy performing neural network and end up with high accuracy. We keep adjusting weights to find a better performance than the initial one, so long as we have enough labeled examples for the network to learn from.  [3]
To training neural networks requires enormous computational power to deal with huge data set. Our university has 7 NVIDIA 1080Ti graphics cards, we may able to get access to them and use them to train our neural network. Since the project requirement asks the accuracy need to be at least 99.9 %, we need at least 1000 images of each type of seed in order to achieve high accuracy of pure seed. To build our data set, we can get pictures from the previous team’s database. If the data from the previous team is not enough, we will take more images of each type of seeds to adjust the neural network to make it performs better.
We need to pick a framework to implement our neural network. The advantage of using a framework to create a neural network is that it can perform better and make neural network understandable and maintainable for others. Many choices are free online, such as Chainer, Gluon, Tensorflow, and PyTorch. The tool we decided to use to implement the neural network is called TensorFlow.
From the introduction of TensorFlow application, "TensorFlow is an open source software library for numerical computation using data flow graphs. Nodes in the graph represent mathematical operations, while graph edges represent multi-dimensional data arrays (aka tensors) communicated between them. The flexible architecture allows you to deploy computation to one or more CPUs or GPUs in a desktop, server, or mobile device with a single API." After we install TensorFlow in our system, we can use it to make a neural network model.   
By using TensorFlow to train our neural network, there are basically 7 steps we need to follow: 1. Import the images of seeds dataset 2. Explore the data 3. Preprocess the data 4. Build the model (Setup the layers, compile the model) 5. Train model 6. Evaluate accuracy. To import dataset, we use TensorFlow import data and load the data from the database directly, the types of seeds should be labeled. To explore the data, we need to see how many images we have in the database and what is the resolution of each image. To preprocess the data, we need to verify that the data is in the correct format. To build the model, we need to set up the layers and compile the model. The first layer is to transform the images from 2d-array to 1 d-array, and other layers consisted of neurons. Then we compile the model. To compile the model, we need loss function, optimizer, and metrics. The loss function is to measure the accuracy of the model. Optimizer updates model base on the result from loss function. Metrics is used to observe the process of training and testing. After the model is built, we start to train the model. We import data input model, then the model will learn to associate images and labels. For example, the model will be able to decide the type of seed that in the image. To do evaluate accuracy test, we base on the performant of our model and give a test accuracy. [4]

\end{document}
