\documentclass{article}
\usepackage[utf8]{inputenc}

\title{Progress of weeks}
\author{zhanhaox }

\begin{document}

\maketitle

\section{Weeks}
\subsection{On week 4, we met our client on Thursday in the OSU seed lab. Our client curry showed us the equipment of observing seeds which consist of a microscope and a conveyor. Our client told us they hire people to sit there for hours to separate target seed and off-type seed, they need computer science student to design and complete a program that at least can identify the off-type seeds and located them. After the program identified the off-type seed, it will send a signal to stop conveyor or turn the LED light on, then somebody will go to pick these off-type seeds. Then the client had a short presentation for us to get know the previous research of the seed project and introduced what the previous team had done and what are the plans that our current team can choose to take and implement.}

\subsection{On week5, our group works on the client requirements document. The requirements document briefly describes the mission we are working on and the final prototype of the project. Since our group had great meeting with our client, and we went through what exactly the client wants and the plan of doing the project, we basically know how to complete the requirements document that can describe our project in detail. Our client uploaded the new requirements for our project, and he posted it on Slack. The new requirement increases the number of test seed from 2500 seeds to 25000 seeds and the test also need to be complete in 30 minutes. The speed of vibratory conveyor carry seed will around 14 seeds/second, which requires the program to be efficient.  Since the 25000 test seeds only contains 5 or 10 or 25 or up to 50 seeds, the accuracy of pure seed need improves to 99.9 percentage instead of 99.5 percentage. Because the if the accuracy still be 99.5, it means there will be at least 125 off-type seeds go into good seed pile, but the test seeds don’t even include that many off-type seeds. That is why improve the standard of accuracy is necessary in the lasted requirements.}

\subsection{On week 6, our group completed our team standard document. Our plan of work completion is to try to meet scrum tickets/tasks if unable to give a notification at least a day forehead, and any failure requires an explanation. For work quality, all mission from work needs to meet requirements, and the process of work need to be recorded and documented. The document needs to be as professional as possible, and the client requires access to these documents.}

\subsection{On week 7, during class time, we conducted a peer review of the technology review draft. For doing the peer-view, each of students needs to review two of others’ paper and leave comments. Each student needs to give a rating to the articles they have seen. After the peer-review is over, it is necessary to record your name and the name of the author you have seen. The record will be used to count attendance of the class. }

\subsection{On Week 8, the instructor released the grade of tech view paper for students. The grading was restricted. The instructor wants students to remember the mistakes they made by deducting points of these mistakes. However, the instructor also gave an opportunity for students to let them resubmit their tech review document to earn points back. Students have to find their errors from rubric and current them to improve their scores. For my tech review paper, I have affirmed that my idea of technologies is feasible in my project, but the lots of vocabularies that I used in my writing are inaccuracy, some syntax is wrong, and the format of the article is not standardized. These are the mistakes that cause me lost so many points, so I have to correct these to earn my points back.}

\subsection{In week 9, our team didn’t make much progress in the project because of the Thanksgiving holiday. The main task of each team member is to write progress update document in the current week. The article asks us to recap and summarize the goals and purpose of the project we currently work on. In the article, we are required to describe the progress of the project, the problems that were solved or unsolved in details. In this article, each of member in the team writes his part of the article and then our team will combine the articles from all members into an entire and make changes and improvements. Each team member will then do presentation according to what they have written. Our team has five members, and we need to provide 30 minutes video. The mission needs to be assigned to members fairly, so each of our team member requires to give about 6 minutes of video, and then edit these videos together.}

\end{document}
