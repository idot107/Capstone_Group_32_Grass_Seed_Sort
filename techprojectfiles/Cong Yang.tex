\documentclass[10pt,draftclsnofoot,onecolumn,journal,compsoc]{IEEEtran}
\usepackage[margin=0.75in]{geometry}
\usepackage[dvipsnames]{xcolor}
\usepackage{graphicx}
\usepackage{caption}
\usepackage{hyperref}
\usepackage{enumerate}
\usepackage{amssymb}
\usepackage{indentfirst}
\title{Problem Statement}
\author{
  \IEEEauthorblockN{Cong Yang} \\
  \IEEEauthorblockA{CS 461 F2018 \\ Oregon State University}
}

\begin{document}
\maketitle
\newpage
\tableofcontents
\newpage

\newpage
\begin{abstract}
   It is obvious that if we want to discriminating the pure seeds from all other seeds by human, it is not efficient at all. What’s more, if we want to do it fast, the rate of error will be high, so that we can’t achieve our purpose well. The goal of making this project is discriminating the pure seed form all other weed seed, crop seed, and other plant material by using a high resolution camera, a computer vision system, machine learning and a cloud-based or machine classifier. This project will be a form of app connected and controlled by computer. On the computer, user will be able to check to status of discrimination and control the running status. In this project, I think the status of weeds will be a good criteria, such as weight, color, wrinkle, humidity, and so on.
\end{abstract}

\newpage
