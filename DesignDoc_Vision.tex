\subsection{context}

The goal of the vision processing subsystem is to transform each image received by the camera, and tranform the result into several
individual images, one for each identified seed, so that the neural net can classify them. 
This system will be responsible for identifying each seed,
and creating the image data for each.

\subsection{Composition} 

This system will be composed of a single algorithm, making use of the OpenCV image processing library.
By making use of OpenCV, the algorithm should be reduced to initial filtering, noise removal, and thresholding.

\subsection{Algorithm}
Filtering will take the initial image and create a boolean array to establish whether each pixel made it through the filter.
Since the background will be quite consistant, this filter should be easy to create, but will be more formally established during testing.
After filtering we remove noise removal, which can be accomplished easily using the openCV morphologyEx() MORPH_OPEN operation.
Finally, utilizing OpenCV's distanceTransform() and threshold() functions, we can easily find the coordinates of each seed, and 
create seperate images for each.

\subsection{Logical}

This subsystem will have a single function facing the remainder of the project. 
This function will take a single image as a parameter, and return an array containing
a set of images, each with one of the seeds identified. 

\subsection{Dependancies} 

This subsystem will require image data recieved from the camera subsystem, and is necessary to
provide images to the neural network.

\subsection{Resources} 

The vision processing subsystem will run on the rasberri pi, 
and requires no additionaly memory resources.



