
\subsection{Context}

The goal of the vision processing subsystem is to transform each image received by the camera, and tranform the result into several
individual images, one for each identified seed, so that the neural net can classify them. 
This system will be responsible for identifying each seed present on the received image,
and creating a smaller image containing just that seed for each of them.

\subsection{Composition} 

This system will be composed of a single algorithm, making use of the OpenCV image processing library.
By making use of OpenCV, the algorithm should be reduced to initial filtering, noise removal, and thresholding.

\subsection{Algorithm}
Filtering will take the initial image and create a boolean array to establish whether each pixel made it through the filter.
OpenCV supplies a threshold() function which can easily apply such a filter, but testing will be required to identify the exact parameters required to seperate the seeds
from the background.
Since the background will be quite consistant, this filter should be easy to create, but will be more formally established during testing.
After filtering we remove noise removal, which can be accomplished easily using the openCV morphologyEx() MORPH\_OPEN operation.
w
Finally, utilizing OpenCV's distanceTransform() function and another filter, we can ensure each identified block is seperate, and use OpenCV's connectedComponents() function to identify each block and its coordinates.
With those coordinates we can easily create a smaller image containing a single seed based on the initial image.

\subsection{Logical}

The vision processing subsystem will have a single function facing the remainder of the project. 
This function will take a single image as a parameter, and return an array containing
a set of images, each with one of the seeds identified. 

\subsection{Dependancies} 

The vision processing subsystem will require image data recieved from the camera subsystem.

\subsection{Interactions}

The vision processing subsystem will interact with both the camera, and neural network subsystems. 
The vision processing system serves as the in between code necessary to translate the image received from the camera
into images that the neural network is capable of identifying.

\subsection{Resources} 

The vision processing subsystem will run on the rasberri pi, 
and requires no additionaly memory resources.

\subsection{Rational}

It is not feasable for the neural network to classify the entire image at once, so we need a method to seperate each seed, and feed images to the neural network individually. 
OpenCV provides one of the most usable code libraries for image processing, and was identified as the best technology to use by our team.
The vision processing algorithm has the potential to be quite complicated, so keeping it seperate from the other systems is important for maintainability.
 



