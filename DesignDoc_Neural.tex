\subsection{Context} Our neural network is the heart of the project, it allows us to process quickly and accurately classify images that are fed through it. We will be feeding training images from the hard drive to train it, but it will spend most of its life in deployment on the Movidius Neural Compute stick. Which will be plugged into the raspberry pi.
 \subsection{Composition} We will build the neural network using Keras which is a wrapper library for the popular tensorflow library. This will allow us to quickly test and alter code. 
\subsection{Structure} The structure of the neural network will be a convolutional neural network optimized for image processing. 
\subsection{Logistics and Dependencies} Because the neural network will interface our other devices and programs through the raspberry pi we need to make sure that the pi has Tensorflow and Keras installed and all the required dependencies including Numpy, Pillow, Scipy, Jupyter, matplotlib and scikit-learn. 
\subsection{Algorithms} There are a myriad of algorithms internally within CNNs. We can pick and choose between different convergent optimizing functions however we will use Adam optimizer. Furthermore, there is pooling, convolutions, different activation functions, ReLu, loss functions, dropout, and decays. All of these factors will play a role into how effective our algorithm will be as a whole, and it will be up to us to fine tune the overall architecture. 
\subsection{Referencess}
[1]Nielsen, \& A., M. (1970, January 01). Neural Networks and Deep Learning. Retrieved from http://neuralnetworksanddeeplearning.com/chap1.html \newline
[2]Shaikh, F., \& Faizan. (2018, April 04). An Introduction to Implementing Neural Networks using TensorFlow. Retrieved from https://www.analyticsvidhya.com/blog/2016/10/an-introduction-to-implementing-neural-networks-using-tensorflow/ \newline
[3](2017, November 27). How do we 'train' neural networks ? – Towards Data Science. Retrieved from https://towardsdatascience.com/how-do-we-train-neural-networks-edd985562b73 \newline
[4]Maladkar, K. (2018, November 20). 6 Types of Artificial Neural Networks Currently Being Used in ML. Retrieved from https://www.analyticsindiamag.com/6-types-of-artificial-neural-networks-currently-being-used-in-todays-technology/ \newline
[5]Train your first neural network: Basic classification | TensorFlow. (n.d.). Retrieved from https://www.tensorflow.org/tutorials/keras/basic\_classification \newline
\subsection{Rational} A CNN is necessary because of its ability to apply filters and classify images even if the subject is awkwardly oriented. This is valuable for our project because the seeds being fed through the conveyor belt will be in different locations and orientations. Furthermore, a CNN ensures the ability to distinguish between noise and positive signals continuously due to its lack of a memory module. 
