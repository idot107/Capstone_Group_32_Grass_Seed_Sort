\subsection{Context}

The camera subsystem is responisble for capturing images of seeds on the conveyor belt on a timed interval and sending the images to the rasberry pi for image processing. The camera subsystem needs to provide clear and quick images to the vision processing subsystem, so that the seeds captured by each image can be identified and classified by the image processing subsystem. 

\subsection{Composition}
This subsytem will make use of a camera, usb connector, image control program, and a constant light. The camera needs to take high quality images fast enough to capture 15 or more small grass seeds per second. 

\subsection{Structure}
The image control program will need to send a signal repeatedly with a set interval in order to capture the images at a speed that is effective and useful for our system. The usb connector will be used to send signals and data in and out of both the camera as well as the raspberry pi. These images will then be processed by a different subsystem. The constant light will need to provide a source of light to the conveyor belt so that the images used dont change even with outside sources of light.

\subsection{Rational}
In order to get high quality images the team decided that a resolution of 1920 x 1080 would be sufficient for high definition quality to determine good and bad seeds. For image speed the camera would need to run more than 10 frames per second and 30 frames per second will allow us to be well above what we need. The camera we decided to use for our project is the Logitech C920 Pro Webcam which is able to take images at 1920 x 1080 resolution as well as doing so at 30 frames per second. This camera also comes with mounts attached to it and the mount allows the camera lens to look up or down for adjusting. 

The next part of this component is the program that sends a take image signal to the camera and saves the image in a specified way. The program will use a timer to take an image, wait, then take another image, wait, repeat. This will allow for a controlled amount of images sent as well as capturing all the seeds that go by. In order to do this OpenCV is an open source library for computer vision in python that is capable of taking images and processing them. Since Python is the code that we are planning on using for connecting our system OpenCV is perfect for doing image taking and processing. 

The light that we have will need to be strong enough to reduce the effect of outside light sources that change throughout the day. Since the program could be running at any time, any outside variables need to be removed in order to get consistant and accurate results. The light would need to be mounted above the seeds near or next to the camera in order to reduce shadows. For this we will use a generic desk lamp since it is able to be moved and adjusted and can give consistent lighting.

\subsection{Logical}
For the camera subsystem we will be using OpenCV, an open source vision software, in order to capture and analyse initial images. The program writen with OpenCV will send a call to the camera to take an image every interval of time. The interval that we use to take images will be faster than the processing subsystem that way it doesn't have to wait for a return signal to take the next image. This will streamline the image taking and updating process so that it is not dependent on the speed of the next subsystem.

\subsection{Interaction}
The camera control program will interact with the camera, the hard disk, and the image processing system. The way it interacts is by sending a signal to the camera, then saving the image back to the hard disk, and then the image processing system will take the image from the hard disk and process it. The control program will then continue to to send signals on the timed intervals.

\subsection{Resources}
The camera system requires only a small amount of processing and storage. It needs enough storage to save a single high definition image that the camera takes and processing power to conisitently send a signal to keep taking images. It also needs to use one usb port on the Raspberry pi to connect the camera.
