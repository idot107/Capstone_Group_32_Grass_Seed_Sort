\documentclass[onecolumn, draftclsnofoot,10pt, compsoc]{IEEEtran}
\usepackage{graphicx}
\usepackage{url}
\usepackage{pgfgantt}
\usepackage{minted}
\usepackage{setspace}
\usepackage{geometry}
\geometry{textheight=9.5in, textwidth=7in}

% 1. Fill in these details
\def \CapstoneTeamName{		The Seed Team}
\def \CapstoneTeamNumber{		32}
\def \GroupMemberOne{			Bharath Padmaraju}
\def \GroupMemberTwo{			Kevin Deming}
\def \GroupMemberThree{			Haoxuan Zhan}
\def \GroupMemberFour{			Cong Yang}
\def \GroupMemberFive{			Christopher Wohlwend}
\def \CapstoneProjectName{		Pure Grass Seed Sorter}
\def \CapstoneSponsorCompany{	Oregon State University Seed Lab}
\def \CapstoneSponsorPerson{		Dan Curry}

% 2. Uncomment the appropriate line below so that the document type works
\def \DocType{		%Requirement Document
				%Requirements Document
				%Technology Review
				%Design Document
				Progress Report
				}
			
\newcommand{\NameSigPair}[1]{\par
\makebox[2.75in][r]{#1} \hfil 	\makebox[3.25in]{\makebox[2.25in]{\hrulefill} \hfill		\makebox[.75in]{\hrulefill}}
\par\vspace{-12pt} \textit{\tiny\noindent
\makebox[2.75in]{} \hfil		\makebox[3.25in]{\makebox[2.25in][r]{Signature} \hfill	\makebox[.75in][r]{Date}}}}
% 3. If the document is not to be signed, uncomment the RENEWcommand below
\renewcommand{\NameSigPair}[1]{#1}

%%%%%%%%%%%%%%%%%%%%%%%%%%%%%%%%%%%%%%%
\begin{document}
\begin{titlepage}
    \pagenumbering{gobble}
    \begin{singlespace}
    	\includegraphics[height=4cm]{coe_v_spot1}
        \hfill 
       % 4. If you have a logo, use this includegraphics command to put it on the coversheet.
        %\includegraphics[height=4cm]{CompanyLogo}   
        \par\vspace{.2in}
        \centering
        \scshape{
            \huge CS Capstone \DocType \par
            {\large\today}\par
            \vspace{.5in}
            \textbf{\Huge\CapstoneProjectName}\par
            \vfill
            {\large Prepared for}\par
            \Huge \CapstoneSponsorCompany\par
            \vspace{5pt}
            {\Large\NameSigPair{\CapstoneSponsorPerson}\par}
            {\large Prepared by }\par
            Group\CapstoneTeamNumber\par
            % 5. comment out the line below this one if you do not wish to name your team
            \CapstoneTeamName\par 
            \vspace{5pt}
            {\Large
                \NameSigPair{\GroupMemberOne}\par
                \NameSigPair{\GroupMemberTwo}\par
                \NameSigPair{\GroupMemberThree}\par
                \NameSigPair{\GroupMemberFour}\par
                \NameSigPair{\GroupMemberFive}\par
            }
            \vspace{20pt}
        }
        \begin{abstract}
        % 6. Fill in your abstract    
        	The primary objective of the project is to automate grass seed sorting. The members of the group will be building software to be able to discriminate between pure grass seeds from all other plant seeds including but not limited to weeds, and crop seeds. The method we will utilize will be a combination of implementing computer vision and deep learning algorithms to accurately identify off type seeds under a high definition camera. This will vastly reduce the stress and workload imposed upon seed analysts, and likely speed up the sorting process. Not only does this project offer a opportunity to improve seed research, but also creates possibilities in other fields where our technology can automate menial and repetitive tasks.
        \end{abstract}     
    \end{singlespace}
\end{titlepage}
\newpage
\pagenumbering{arabic}
\tableofcontents
% 7. uncomment this (if applicable). Consider adding a page break.
%\listoffigures
%\listoftables
\clearpage

\section{Purpose}
Our project purpose is to reduce the total amount of seed that seed analysts are required to sort by hand by automating a large portion of the process. Our hope is to take large quantities of seeds and sort out between 80-90 \% of the work.

\section{Current Progress}

Currently we have completed documentation of the project requirements and our initial design. As seed proceeds down a conveyor belt, we plan to utilize a camera, and machine learning techniques to identify off-type seeds, and separate them from the pure seed. With a clear design, and timetable, we have a clear plan for the remainder of the year. 

\section{Goals}
Next term we plan to immediately begin collecting training data from the seed lab. As thousand's of images are necessary, our group will have to rotate data collection shifts. Moreover, we will start developing the database systems and acquiring/testing our hardware. We will then need to implement our training pipeline in order to train our neural network. We will then integrate the neural network into our final product that will be attached to the conveyor belts. Alpha and Beta testing will be done in real time shortly after our prototype is deemed successful in initial trial runs. Long term goals for the project is to be able to scale the system up to more machines in order to parallelize the process and yield more output. 

\section{Problems Faced}
Organization and communication between a group is always a difficult barrier to get through especially if the members of the group don't know each other. However our group is very passionate about the project and friendly so we were able to familiarize ourselves with each other and create a productivity pipeline using Trello, Slack, GitHub, and Google Drive. Another problem we faced was scope of project. It turns out that the mechanical part of the project will done by a mechanical engineering capstone group. We had to also keep their design in mind when developing our software designs. 

\section{Weekly reports}

\subsection{Week One}
Groups weren't formed at this point, but as individuals, we looked at each of the available projects and oriented ourselves.
\subsection{Week Two}
Groups still weren't formed, but we made final group selections at the end of this week.
\subsection{Week Three}
Our first team meeting happened, and we set up a meeting with our client, and discussed the experience each of us had already. We set up our github page, but had some troubles adding instructors and TAs to the github page. This has since been rectified.
Each of us also completed our individual problem statements.
\subsection{Week Four}
This week we combined our problem statements into our group problem statement. This process was essentially just pulling paragraphs from each of our individual documents, and combining the best ones.
We also had our first client meeting where we learned about previous similar projects the client had requested and some suggested mechanical prototypes that we could design our software to operate on. Notably, off-type seed in many cases looks very similar, but is still distinguishable.
\subsection{Week Five}
This week we did little as a team, as midterms were in full swing. We did meet briefly to assign each team member sections of the requirements document to complete, and begin assembling a list of potential topics for the technology review. 
\subsection{Week Six}
This week was particularly busy, as we each completed our sections of the requirements document, and spent some time assembling those sections, and reviewing each others sections. After producing a final list of topics for the technology review, they were distributed, and each of us researched those technologies for our tech review.
\subsection{Week Seven}
Tech reviews were turned in, and peer reviews were completed. Not much else was accomplished this week.
\subsection{Week Eight}
This week we did some minor updates to past documentation based on Client and TA feedback. We had not yet created an outline for the Design document, but had plans to do so.
\subsection{Week Nine}
This week being thanksgiving break, we made little progress, but did finally manage to create an outline for the design document based on the design we had already discussed. 
\subsection{Week Ten}
This week we Completed the tech review. We procrastinated a lot, but spent the time to create a solid document, and since all of us had a solid understanding of our design from previous discussions, we were able to complete a quality document. We made plans to complete this progress update, and met with the client in order to get sign off on all of our documentation. 

\section{Retrospective}

\begin{center}
\begin{tabular}{
|p{0.1\linewidth}
|p{0.25\linewidth}
|p{0.25\linewidth}
|p{0.25\linewidth}|
}
\hline
Project & Positives & Deltas & Actions \\\hline
Problem Statement & We worked efficiently, and assembled the document from each team members sections. & Because we met with the client so late, we had little time to lay out the problem as completely as desired due to lack of information.  & More frequent meetings with the client will give us access to the most recent updates and information more frequently. Because five schedules is a lot to coordinate, at times, we may meet with fewer members in order to achieve this. \\\hline
Requirements Doc & We laid out the needs of the project well, and we have clear metrics for success.  & We went into a little to many details on implementation. & Revisions should be made to ensure that the implementation details are removed leaving only the project requirements.  \\\hline
Tech Reviews  & As a team, we managed to review many different technologies, including databases, cameras, and GPU's, giving us a strong foundation to build one. & We had trouble communicating requirements and individual tasks for the tech review. & Posting clear instructions, structure, and tasks in an easy to see and access place. \\\hline
Design Doc & Despite procrastinating we managed to describe all aspects of our design in a way that was clear to us, and our client. We have a clear plan going forward & Some typos and unhelpful diagrams were present in this document, in part because of our rush to complete it. Additional information describing how we plan to train the network is also desired. & Taking TA, and client feedback, many of these sections will be revised as we update the design document moving forward. Making an effort to establish weekly meetings will also help prevent future procrastination less likely in the future. \\\hline
\end{tabular}

\end{center}



\end{document}
